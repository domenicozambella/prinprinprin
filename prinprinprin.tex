\documentclass[10pt]{article}
\usepackage{amsrefs}
\usepackage[colorlinks=true,
            bookmarksopen=false,
            linkcolor=blue,
            citecolor=red
            ]{hyperref}
\title{Prin prin prin}
\author{Domenico Zambella}
\begin{document}
\maketitle
\section{State of the art}
Keisler measures where introduced by Jerome Keisler in a seminal article~\cite{MR890599} back in 1987.
Yet they went almost unnoticed for the best part of three decades until Hrushovski, Peterzil, and Pillay used them to settle the Pillay conjectures on definably compact groups definable in o-minimal theories --a major open problem in \textit{model theory\/} and \textit{Lie group theory\/}~\cite{MR2373360}.

Since then, Keisler measures have become an ubiquitous tool in both pure and applied model theory.
Initially they played an important role in the context of NIP theories (theories without the indepence property, also known as dependent theories)~\cite{MR3560428}.
For example they are a crucial tool in the study of definably amenable groups definable in NIP theories~\cite{MR3787403}, \cite{MR2427062}.
But they are more and more frequently applied outside this context, e.g.\@ when only a single formula (not the whole theory) is NIP~\cite{MR4010500}, \cite{arXiv:2103.09946} or \textit{in the wild\/} i.e.\@ with no assumption of tameness~\cite{arXiv:2103.09946}.
In fact, Keisler measures have interesting application in finite and pseudofinite combinatorics and allow us to generalize combinatorial results in a natural measure theoretic framework~\cite{MR4222408}, \cite{MR3666030}.

Keisler measures also play a role in explaining important dividing lines between classes of theories.
A pervasive theme in model theory is the discovery of basic combinatorial principles that saparete theories (or single formulas) that are tame/well-behaved/feasible/learnable from those that are not.
One recent dividing line, that of textit{distal theories}, was recently introduced by Pierre Simon~\cite{MR3001548}.
This dividing line can be characterized in different ways: using model theoretical tools (indiscernible sequences); finite combinatorial tools; but also by measure theoretic notions~\cite{MR3335415}.
Each of these perspectives contributes to a deeper understanding of the phenomenon.
A recent influential success of the notion of distality has been obtained by Chernikov and Starchenko~\cite{MR3852184} (see also~\cite{MR3503725}).
They proved that distality induces strong regularity phenomena in graphs, hence that the relevance of the distiction distal/non distal has relevance in different areas of mathematics.

\section{Descrizione di metodi, obiettivi, risultati, diffusione}

\subsection{Randomization}
Keisler measures are finitely additive measures on some algebra of definable sets usually of a saturated model.
Keisler measures generalize the notion of type. 
In fact, a complete type (i.e.\@ a set of formulas) may be viewed as a Keisler measure that assigns measure $1$ to sets definable by a formula in the type, and measure $0$ to all other definable sets.
Types and measures form a syntactic, respectively measure-theoretic apparatus that we superimpose to the model. 
For this reason, they escape the most common model theoretic techniques.
Fortunately, in the case of types, there is an easy remedy.
By the compactness theorem, every type can be realized in a sufficiently saturated elementary superstructure.
Therefore types are natuarally identified with the (external) elements that realize them.
These external elements are tractable with common model theoretical tecniques.

Measures elude this approach. 
For example, the common intuition is that the so called \textit{smooth measures\/} correspond to realized types, but this does not help in interpreting them as elements of a first-order model.
A possible way out is to move away from classical logic. 
Keisler and Ben Yaacov propose to interpret measures in continuous logic as types over a randomized structure~\cite{MR2561997} (the notion of randomization of a structure was previously and independently introduced by Keisler in~\cite{MR1680650}).
This approach is definitely sound, but it adds many mathematical subtleties and notational hurdles to the descripion.
This is why few people have embraced this approach.

\subsection{Loeb samples}

The alternative approach that we propose is low tech and is inspired by a classical idea of Loeb~\cite{MR1680650} widely used in nonstandard analisis.
Every probability measure can be interpreted as a relative frequence evaluated on a sample of hyperfinite size.
Classical first-oder logic is more suited to formalizing samples and frequences than real-valued set function.




\subsection{}





\section*{References}
\begin{biblist}[]\normalsize
\addcontentsline{toc}{section}{References}

\bib{MR2561997}{article}{
   author={Ben Yaacov, Ita\"{\i}},
   author={Keisler, H. Jerome},
   title={Randomizations of models as metric structures},
   journal={Confluentes Math.},
   volume={1},
   date={2009},
   number={2},
   pages={197--223},
   issn={1793-7442},
   doi={10.1142/S1793744209000080},
}
\bib{MR3335415}{article}{
   author={Chernikov, Artem},
   author={Simon, Pierre},
   title={Externally definable sets and dependent pairs II},
   journal={Trans. Amer. Math. Soc.},
   volume={367},
   date={2015},
   number={7},
   pages={5217--5235},
   issn={0002-9947},
   doi={10.1090/S0002-9947-2015-06210-2},
}
\bib{MR3787403}{article}{
  author={Chernikov, Artem},
  author={Simon, Pierre},
  title={Definably amenable NIP groups},
  journal={J. Amer. Math. Soc.},
  volume={31},
  date={2018},
  number={3},
  pages={609--641},
  issn={0894-0347},
  doi={10.1090/jams/896},
}
\bib{MR3852184}{article}{
   author={Chernikov, Artem},
   author={Starchenko, Sergei},
   title={Regularity lemma for distal structures},
   journal={J. Eur. Math. Soc. (JEMS)},
   volume={20},
   date={2018},
   number={10},
   pages={2437--2466},
   issn={1435-9855},
   doi={10.4171/JEMS/816},
}
\bib{MR4010500}{article}{
  author={Gannon, Kyle},
  title={Local Keisler measures and NIP formulas},
  journal={J. Symb. Log.},
  volume={84},
  date={2019},
  number={3},
  pages={1279--1292},
  issn={0022-4812},
  doi={10.1017/jsl.2019.34},
}
\bib{arXiv:2103.09946}{article}{
  author={Gannon, Kyle},
  title={Sequential approximations for types and Keisler measures},
  date={2021},
  pages={1--27},
  note={Available at \href{https://arxiv.org/abs/2103.09946}{arXiv:2103.09946}},
}
\bib{arXiv:2103.09946}{article}{
  author={Gannon, Kyle},
  author={Hanson, James},
  author={Conant, Gabriel},
  title={Keisler measures in the wild},
  date={2021},
  pages={1--54},
  note={Available at \href{https://arxiv.org/abs/2103.09137}{arXiv:2103.09137}},
}
\bib{MR2373360}{article}{
  author={Hrushovski, Ehud},
  author={Peterzil, Ya'acov},
  author={Pillay, Anand},
  title={Groups, measures, and the NIP},
  journal={J. Amer. Math. Soc.},
  volume={21},
  date={2008},
  number={2},
  pages={563--596},
  issn={0894-0347},
  doi={10.1090/S0894-0347-07-00558-9},
}
\bib{MR890599}{article}{
  author={Keisler, H. Jerome},
  title={Measures and forking},
  journal={Ann. Pure Appl. Logic},
  volume={34},
  date={1987},
  number={2},
  pages={119--169},
  issn={0168-0072},
  doi={10.1016/0168-0072(87)90069-8},
}
\bib{MR1680650}{article}{
   author={Keisler, H. Jerome},
   title={Randomizing a model},
   journal={Adv. Math.},
   volume={143},
   date={1999},
   number={1},
   pages={124--158},
   issn={0001-8708},
   doi={10.1006/aima.1998.1793},
}
\bib{MR390154}{article}{
   author={Loeb, Peter A.},
   title={Conversion from nonstandard to standard measure spaces and
   applications in probability theory},
   journal={Trans. Amer. Math. Soc.},
   volume={211},
   date={1975},
   pages={113--122},
   issn={0002-9947},
   doi={10.2307/1997222},
}
\bib{MR2427062}{article}{
  author={Onshuus, A.},
  author={Pillay, A.},
  title={Definable groups and compact $p$-adic Lie groups},
  journal={J. Lond. Math. Soc. (2)},
  volume={78},
  date={2008},
  number={1},
  pages={233--247},
  issn={0024-6107},
  doi={10.1112/jlms/jdn018},
}
\bib{MR4222408}{article}{
  author={Pillay, Anand},
  title={Domination and regularity},
  journal={Bull. Symb. Log.},
  volume={26},
  date={2020},
  number={2},
  pages={103--117},
  issn={1079-8986},
  doi={10.1017/bsl.2020.40},
}
\bib{MR3001548}{article}{
  author={Simon, Pierre},
  title={Distal and non-distal NIP theories},
  journal={Ann. Pure Appl. Logic},
  volume={164},
  date={2013},
  number={3},
  pages={294--318},
  issn={0168-0072},
  doi={10.1016/j.apal.2012.10.015},
}
\bib{MR3560428}{book}{
  author={Simon, Pierre},
  title={A guide to NIP theories},
  series={Lecture Notes in Logic},
  volume={44},
  publisher={Association for Symbolic Logic, Chicago, IL; Cambridge
  Scientific Publishers, Cambridge},
  date={2015},
  pages={vii+156},
  isbn={978-1-107-05775-3},
  doi={10.1017/CBO9781107415133},
}
\bib{MR3503725}{article}{
   author={Simon, Pierre},
   title={A note on ``Regularity lemma for distal structures''},
   journal={Proc. Amer. Math. Soc.},
   volume={144},
   date={2016},
   number={8},
   pages={3573--3578},
   issn={0002-9939},
   doi={10.1090/proc/13080},
}
\bib{MR3666030}{article}{
  author={Starchenko, Sergei},
  title={NIP, Keisler measures and combinatorics},
  note={S\'{e}minaire Bourbaki. Vol. 2015/2016. Expos\'{e}s 1104--1119},
  journal={Ast\'{e}risque},
  number={390},
  date={2017},
  pages={Exp. No. 1114, 303--334},
  issn={0303-1179},
  isbn={978-2-85629-855-8},
}
\end{biblist}
  
\end{document}